\section{Introduction}

%\dv{Need to have 6 paragraph. 
%First para stays. \\
%Second should motivate human aware navigation using FMM. This means motivating FMM and then human aware navigation using FMM. Also in this para, mention about global and local path planning strategies.\\ 
%Next para should be about the social cost map using proxemics.(I prefer this order since social cost map can only be mentioned after you explain the framework (FMM or DWA which is the path planner).\\
%4th para seems ok now. We motivated social path planning using social cost maps and planners like FMM or DWA. In this para we talk about how and why uncertainty should be used.\\
%5th para is the place where we talk about our two models.\\
%6th Para should be our results. "We also show from our experiments that...global path planning has these positives and negatives when used for real human aware planning using uncertainty, whereas local strategies have these these issues. }

%\dv{Motivating our work in one para. Looks good for now}
Human-aware navigation is a key problem in social robotics. Navigation is one of the main required functionalities for enabling robots to be actively used in real social environments. %Robust navigation principles incorporating humans in the environment need to be developed. 
Robots have to navigate in environments shared with humans and the quality of their movement strongly influences how their intelligence is perceived~\cite{Althaus2004}. There have been numerous studies on this topic, and conventionally, comfort, naturalness and scalability, have been the main focus of human-aware navigation techniques~\cite{Kruse2013}. However, when the robots are deployed in real complex environments, a key assumption of many of these methods, which is having perfect information about the position of the people, is no longer valid. In this work, we attempt to model one essential aspect of human-aware navigation which has been overlooked in this area, uncertainty in the perception.

%\dv{How do we do human aware navigation. We do it by using social costmaps fed into a navigation planner. So first talk about the path planning, I think we should say that people have tried to do this using global planners. And also talk about local planners. Add a para here about social path planners.}

One important concept used in numerous studies~\cite{Mumm2011,Takayama2009,Walters2011,ferrer2013robot} in this area, is the virtual space around a person that is mutually respected by other humans, called \textit{proxemics}~\cite{Hall1969}.
Based on this concept, depending on the relationship and the interaction that exists between humans, people choose different social distances relating to intimate, personal, social or public contexts.
%Changes in the expected distance may indicate dislike if it is too large or cause discomfort if it is too small.\dv{avoid ideas like these. This has no relevance to our story} 
Social costmaps are a common way to model this principle that and been used in various studies in the field. %\cite{svenstrup2010trajectory}. 
Many factors can be considered for shaping a social costmap, such as age and gender \cite{dautenhahn2006may}, velocity of the motion \cite{kirby2009companion}, etc. However, the proxemics distance has been the main factor when accounting for comfort in the literature.% Other factors such as speed of a person's movement, gender, age, etc. have also been considered in the literature, but are much less common.%\dv{This way of phrasing the available works, makes it feel like uncertainty is just another factor that people ignored to consider. we need to rephrase this idea of social cost maps} 

Navigation exhibits human-awareness by modifying the plans given to the robot as the result of adopting socially-aware path planners.
%A good model of the social costs leads to providing more accurate information to the path pl
Such a planner should take into consideration individual people and possible social interactions taking place between them, when computing the optimal path. This information is provided through perceptual data and is translated to social costs using models such as social costmaps. However, perception will never be perfect and is affected by various elements such as the robot movement, movement of the people, complexity of the environment in terms of occlusions, etc. 


Due to the approximate nature of the models and the less than perfect detectors available, we often can only provide estimates of the locations of the people with an associated uncertainty. Thus, any planning algorithm relying on real perception data, must be able to succeed using this less than perfect estimates. Despite this fact, the assumption of having perfect information about the position of people at all times, is common in the state-of-the-art research in this area% \dv{must add references}
, where the main focus is on the planning itself. However, moving to real applications, poses serious challenges in terms of noisy perception information and high uncertainties, that need to be addressed and modelled in a human-aware approach.


%Different perception sources for person detection and tracking, have different levels of uncertainty and accuracy in their detections. while, there exist trackers able to perform this task with cm-level accuracy, others have a much larger uncertainty associated to them. If a human-aware approach, could take this into account, the same planning method could easily be reused even when the source of perception changes. As an example while overhead cameras can provide position information of people with good accuracy, when moving to on-board perception for a mobile robot, this accuracy and the associated uncertainty will largely change. This is even more significant, if tracking is done using an ultra-wide-band (UWB) tag. So, for more robustness and effectiveness, a human-aware navigation method, should be able to handle all these situations without undergoing major changes. This is possible, by adaptively changing the social costs.\dv{This para is kind of redundant. Does not add much value to generalize our work. In discussions and possible extentions we can mention how to generalize our work. Definitely not in introduction}

%However, perception will never be perfect and is affected by various elements such as the movement of the robot, movement of the person, complexity of the environment in terms of occlusions, etc. Unlike other works which choose to ignore this very important aspect of the uncertainty in perception, we propose a model which computes the uncertainty of the location and orientation of the people in an environment. We aim to study, how this factor should influences the social costs used by the path planner and how taking this into account the resulting trajectories will be improved in terms of social acceptability.

%\dv{Discussion of proxemics makes it harder to follow the narrative. Need to restructure the introduction from here on. The story is broken. first two paragraphs motivated human aware navigation and real perception systems. what now? how do you think we should continue the story? - i am unclear on this. how does proxemics fit into the narrative? transition lines are very important which connect the story.}


%\dv{again. this paragraph needs to be worked on. I don't think we should suddenly switch to human robot interaction. thats not what this paper is dealing with. Uncertainity in human aware navigation. thats all we care about.}
%As stated in~\cite{gomez2013social}, recent work in the human-robot interaction domain can be divided in three categories of 1) human-robot proxemics, 2) human-aware planning and navigation, and 3) robot-to-human approaching and behaving. In this work we focus mostly on the second category using the concepts from human-robot proxemics. %Human-aware path planning has been an area of interest particularly in the recent years. 

In this work we propose a model that incorporates the uncertainty of position estimates of people into the social costmap. We aim to study, the effect of this factor on social costs and how taking this uncertainty into account in the extended model, can result in trajectories that are improved in terms of social acceptability. The proposed human-aware navigation system, uses uncertainty-based social costs along with a Fast Marching Method (FMM)-based~\cite{sethian1999fast} path planner for achieving socially-aware plans. 


The model is tested in a stochastic environment with varying perception uncertainty, in an extensive set of experiments. Six evaluation metrics have been introduced and the results of applying them to the proposed methods have been reported. Real robot experiments show how accounting for uncertainty of perception has improved the robot trajectories in terms of social distance and naturalness, as the complexity of the environment grows.

The remainder of this article is organized as follows. In section~\ref{Literature} we briefly introduce the related works in this area. Section~\ref{perception_model} explains the probabilistic perception model and Section~\ref{social_nav} explains the model of social navigation. The robotic platform 
%~\cite{Messias2014robotic} %used in our tests, 
and the experimental setup will be explained in detail in section \ref{exp}. In section~\ref{res}, we show the results of the real robot experiments and finally, section~\ref{sec:conclusion} concludes this work.

%\paragraph{Story}  - Quantify uncertainity in perception. Use this for generating cost maps adaptively. FMM creates plans using this uncertainity model.



%\dv{Modelling and accurately computing the location and orientation of a person in an environment is a hard problem. Typically, in state of the art people tracking methods, a bayesian approach is used.}


% \paragraph{Motivation and approach} - Perception will never be perfect. Unlike other works which choose to ignore this very important aspect of the uncertainty in perception, we propose a model which computes the uncertainty of the location and orientation of multiple people in an environment. We use this model to make informed choices about the cost function for navigation which is then used by a state of the art navigation model using FMM. We dynamically replan based on the perception information available. The key idea is that in cases where the perception is unreliable, we need to make sure that the robot navigates prioritizing safety over social norms.

% paragraph{Note} Active perception is also important. The robot movement causes uncertainty which can be actively reduced by taking paths which improve the perception information.



%\subsubsection{Additional Stuff}
%However, contrary to approach we propose in this paper, these works are only validated in computer simulation.Comfort is defined as a relative state in which the body is relieved of unpleasant sensory or environmental stimuli [25]




%//-------------------

%To the best of our knowledge, there does not exit any work on including uncertainty of perception in this problem. If the center of the costmap, which indicates the position of the person in not deterministically known, the costmap can not correctly model the social costs and hence the social path planning could fail in finding an appropriate socially compliant path. This problem becomes much more critical in real applications where robustness is vital for succeeding under different conditions.  

%Robots will progressively become part of the work spaces and habitats of humans.Despite where they are located, in home environments or hospitals as assistants, in factories as co-workers or as guides in supermarkets, a key behavior which they all share is navigation.Robots have to navigate in environments shared with humans and the quality of their movement strongly influences  how their intelligence is perceived~\cite{Althaus2004}. Consequently, new methods for robot navigation in the presence of humans are being studied extensively. Kruse et al.~\cite{Kruse2013} present a thorough survey of such methods where they identify comfort, naturalness and scalability as the three key issues addressed by the existing human-aware robot navigation methods so far. %(last 2 sentences taken from  RV's paper)
%Hence socially-aware navigation which pays special consideration to people 


%Presence of humans in an environment should be properly perceived by a robot as a requirement for a socially-aware path planner that takes into consideration individual humans and possible social interactions taking place in the environment. This information can be obtained by an external source such as an overhead camera or can be attained using on-board sensors of the robot. while there are advantages and disadvantage to both cases, on-board perception allows for a fully autonomous and independent robot that relies only on itself for taking decisions about the world. Moreover, there are environments which do not permit the use of external infrastructures required for providing the requisite perception, due to reasons such as privacy or security issues or etc. Therefore, providing the required perception for human-aware navigation with on-board sensors is a desirable characteristic for a robot which helps integration of such systems into real-world environments.  \dv{This paragraph is irrelavant to motivating our work. lets stick with only the story of uncertainity in perception}

%Presence of humans in an environment should be properly perceived by a robot as a requirement for a socially-aware path planner that takes into consideration individual humans and possible social interactions taking place in the environment. However, perception will never be perfect and is affected by various elements such as the movement of the robot, movement of the person, complexity of the environment in terms of occlusions, etc. Due to the approximate nature of the models and the less than perfect detectors available, we often can only provide estimate of location of the people with significant uncertainty. Any planning algorithm which needs to rely on real perception sources must be able to use this less than perfect estimates. The assumption of having perfect information about the position and orientation of people at all times, is common in the state of the art research in this area, where the main focus is on the planning itself. However, moving to real applications, poses serious challenges in terms of noisy perception information and high uncertainties, that need to be addressed and modeled in a human-aware approach.

%\dv{By the time we are here, we should get the idea about social cost map clear, socially aware path planner clear. then the next para makes sense.}