\section{PROBABILISTIC TRACKING MODEL}
We are interested in the underlying \textit{state} of the environment which is modelled as a tuple: location and orientation of the people. The \textit{discriminative detectors} used to estimate these state variables have associated noise due to various state factors such as occlusion, lighting conditions, different posture of people, motion of the robot and the people. Coupled with this, there is also stochasitcity in the state transitions, which makes it hard to compute an exact estimate of the location of the people. A principled approach to solve this problem, is to compute a \textbf{belief} (posterior distribution) over the states using recursive bayesian estimation. More formally,

LeT $\textbf{X}_{t}$ be the state of the environment at time \textit{t}.
\begin{align}
P(\textbf{X}_{t} | \textbf{O}_{t}) = \dfrac{P(\textbf{O}_{t} | \textbf{X}_{t}) P(\textbf{X}_{t}|\textbf{X}_{t-1})} {P(\textbf{X}_{t-1})}
\end{align} 

Where, $P(\textbf{O}_{t} | \textbf{X}_{t})$ is the likelihood of the state given all our observations (detector outputs). Computation of this likelihood is best performed by using a learnt model of how the detectors perform for different states. Details of how these \textit{observation models} are created is explained in the next section.

$P(\textbf{X}_{t}|\textbf{X}_{t-1})$ is the \textit{transition model} which models the evolution of the state variables. For a multi person natural environment, an exact analyitical model is intractable. We model the motion of the people independently. we assume a first order model for motion incorporating velocity of the person. Interactions between people are environment specific and is outside the scope of this work. So we choose to ignore this aspect of the environment and attribute it to the uncertainity we have in the state estimation. The exact model for a multi person environment is given in section ......

Another important aspect is the computation of this belief itself. the state space is extremely complex so as to compute exactly this probability 
distribution over the states. We use an MCMC based sampling algorithm to approximately compute the belief. The details of our sampling alogorithm is given in section .......

%For the purpose of the human aware navigation planner, we transform these beliefs to a lower dimensional representation which is a 2-dimensional gaussian over the location of the person.

\subsection{Detector}
We use a probabilsitic background based detector
\subsection{Observation models}
We learn from data, the distribution $P(\textbf{O}|\textbf{X})$ for each detector. 

