%In this work, we present a novel approach to human-aware navigation based on the fast marching method, by probabilistically modeling the uncertainty in on-board perception components of a social robotic system and investigating its effect on the overall social navigation performance. We have extended the model of the social costmap around a person to consider this new uncertainty factor which plays an important role in situations with noisy perception. Real robot experiments have been carried out to show the effectiveness of our approach in realistic scenarios with noisy perception involving a single dynamic human. Results show how our approach has been able to achieve trajectories which are more socially-aware compared to the basic navigation approach, and the human-aware navigation approach which relies solely on perfect perception.
In this work, we present a novel approach to human-aware navigation based on the fast marching method, by probabilistically modeling the uncertainty of perception for a social robotic system and investigating its effect on the overall social navigation performance. We have extended the model of the social costmap around a person to consider this new uncertainty factor which plays an important role in situations with noisy perception. Additionally, a variation of the Dynamic Window Approach, which takes social costs into account, has been considered for navigation to discard or penalize velocity candidates that lead to unnatural or uncomfortable movements in the vicinity of humans. 
Real robot experiments have been carried out to show the effectiveness of our approach given noisy perception, in the presence of single/multiple, static/dynamic humans. Results show how our approach has been able to achieve trajectories which are more socially-aware compared to the basic navigation approach, and the human-aware navigation approach which relies solely on perfect perception.