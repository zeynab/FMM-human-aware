\section{Background}
\label{sec:background}
A human-aware navigation system, is comprised of several key components, i.e., perception, social path planning and navigation. In this section we will briefly explain each of these elements in the context of our work.

\subsection{Perception}

Presence of humans in an environment should be properly perceived by a robot as a requirement for a socially-aware path planner that takes into consideration individual humans and possible social interactions taking place in the environment. This information can be obtained by an external source such as an overhead camera or can be attained using on-board sensors of the robot. Different perception sources for person detection and tracking, have different levels of uncertainty and accuracy in their detections, and are affected by various elements such as the movement of the robot, movement of the person, complexity of the environment in terms of occlusions, etc. while, there exist trackers able to perform this task with cm-level accuracy, others have a much larger uncertainty associated to them. 

Talk about deterministic v.s. probabilistic trackers. 
Say we chose omni-directional overhead camera, with which detection and tracking for both cases.Explain briefly the MCMC sampling.

We are interested in the underlying \textit{state} of the environment which is modelled as a tuple: location and orientation of the people. The \textit{discriminative detectors} used to estimate these state variables have associated noise due to various state factors such as occlusion, lighting conditions, different posture of people, motion of the robot and the people. Coupled with this, there is also stochasticity in the state transitions, which makes it hard to compute an exact estimate of the location of the people. A principled approach to solve this problem, is to compute a \textbf{belief} (posterior distribution) over the states using recursive Bayesian estimation. More formally,

Let $\textbf{X}_{t}$ be the state of the environment at time \textit{t}.
\begin{align}
P(\textbf{X}_{t} | \textbf{O}_{t}) = \dfrac{P(\textbf{O}_{t} | \textbf{X}_{t}) P(\textbf{X}_{t}|\textbf{X}_{t-1})} {P(\textbf{X}_{t-1})}
\end{align} 

Where, $P(\textbf{O}_{t} | \textbf{X}_{t})$ is the likelihood of the state given all our observations (detector outputs). Computation of this likelihood is best performed by using a learnt model of how the detectors perform for different states. Details of how these \textit{observation models} are created is explained in the next section.

$P(\textbf{X}_{t}|\textbf{X}_{t-1})$ is the \textit{transition model} which models the evolution of the state variables. For a multi person natural environment, an exact analytical model is intractable. We model the motion of the people independently. we assume a first order model for motion incorporating velocity of the person. Interactions between people are environment specific and is outside the scope of this work. So we choose to ignore this aspect of the environment and attribute it to the uncertainty we have in the state estimation. The exact model for a multi person environment is given in section ......

Another important aspect is the computation of this belief itself. the state space is extremely complex so as to compute exactly this probability 
distribution over the states. We use an MCMC based sampling algorithm to approximately compute the belief. The details of our sampling algorithm is given in section .......

%For the purpose of the human aware navigation planner, we transform these beliefs to a lower dimensional representation which is a 2-dimensional gaussian over the location of the person.

\subsubsection{Detector}
We use a probabilistic background based detector
\subsubsection{Observation models}
We learn from data, the distribution $P(\textbf{O}|\textbf{X})$ for each detector.



\subsection{Social path planning}

Human-aware navigation focuses on the interaction dynamics between humans and robots that occur as a result of navigation~\cite{Kruse2013}. 
In the literature, we can find several strategies for comfort ranging from appropriate approaching strategy~\cite{Dautenhahn2006}, maintaining appropriate distance~\cite{Takayama2009}, control strategies to avoid being noisy~\cite{Martinson2007} and use of planning for avoiding interference~\cite{Vasquez2012}.

In this work we focus on the principle of proxemics which is the most common in the literature of human-aware navigation, with social costs encoded as costmaps similar to ~\cite{gomez2013social}. The personal space around a human can be defined as the mixture of two Gaussian functions, one for the front and another one for the rear part of the area surrounding the person.
A Gaussian function $\phi$, centred on $p$ with covariance matrix $\Sigma$, is defined as follows:

\begin{equation}
\phi(q) = e^{(-\frac{1}{2}(q-p)\Sigma^{-1}(q-p))}
\end{equation}

$q$ indicates the position of a point and $\Sigma$ is:
\begin{equation}
\Sigma = \begin{pmatrix}
{\sigma}_{x}^2  & 0\\ 
 0& {\sigma}_{y}^2 
\end{pmatrix}
\end{equation}

${\sigma}_{x}$ and ${\sigma}_{y}$ are used to modulate the shape of the Gaussian and are traditionally chosen in a way to respect the personal space of a person as indicated by the proxemics principle. Getting closer to a person, will cause an increase in the value of the function, and hence the social cost associated to that position will increase.


If the center of the costmap, which indicates the position of the person in not deterministically known, the costmap can not correctly model the social costs and hence the social path planning could fail in finding an appropriate socially compliant path. This problem becomes much more critical in real applications where robustness is vital for succeeding under different conditions.

Our goal is to show that the assumption of having perfect information about the state of the human is unrealistic and in real situations when the robot has to deal with uncontrolled environments, the uncertainty in this information can not be ignored. We can think of false negative detections where the robot misses to take on person into consideration, false positive detections where other objects are detected as humans, noisy estimations of position, orientation, velocity of the person, etc. To the best of our knowledge, there does not exit any work on including uncertainty of perception in this problem.


 


\subsection*{Navigation}

Global path planning and local path planning are the main components of an autonomous navigation system.
We base our navigation method on FMM \cite{ventura2015} for global path planning and DWA~\cite{fox1997dynamic} for obstacle avoidance and more reactive control. However, these planners will be modified to account for social costs and constraints in their planning. 

FMM computes the optimal path for the robot for a given destination, according to a potential filed created by the setting of obstacles in an environment based on wave propagation principles. In a constantly changing environment a given plan may no longer be valid over time. So applying FMM continuously is one solution to deal with this dynamicity. However, the frequency of this planning should be reasonable given the computational cost of this global planning method. In our system, social costs are incorporated into the FMM method when computing the optimal path by means of imposing the social costs as virtual obstacles.

In the case of DWA, the method has been modified to include a social component. The inherent reactive nature of this method, given the small window of planning which is required for ensuring collision, may not seem to benefit from human-awareness. However, we believe this is an interesting problem to investigate. We will discuss this in more details in section \ref{sec:algorithms}. 




%measures:paper Including Human Factors for Planning Comfortable Paths: Bio-sensing devices , modelling passenger comfort in a real environment to build a path planner. 
%Visibility index: Y. Morales, J. Even, N. Kallakuri, T. Ikeda, K. Shinozawa, T. Kondo, and N. Hagita, “Visibility analysis for autonomous vehicle comfortable navigation,” in Robotics and Automation (ICRA), 2014 IEEE Interna- tional Conference on, May 2014, pp. 2197–2202