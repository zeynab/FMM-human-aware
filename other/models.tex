\documentclass[11pt]{article}
\usepackage{graphicx}
\usepackage{url}
\usepackage{listings}
\usepackage{times}
\usepackage{epsfig}
\usepackage{graphicx}
\usepackage{amsmath}
\usepackage{amssymb}
\usepackage{listings}
\usepackage{verbatim}

\long\def\dv#1{\textcolor{red}{**[DV] #1**}}
\long\def\zt#1{\textcolor{green}{**[ZT] #1**}}

\begin{document}

\title{Human aware navigation - Uncertainity}
\date{\today}
\maketitle


\section{Summary}

\section{Uncertainity model}
%We are interested in the uncertainity over the location of the people. This uncertainity is a function of the state of the world, namely location and velocity of the multiple people in the environment and also the location and velocity of the robot.

\subsection{Principled uncertainity}
We are interested in the underlying \textit{state} of the environment which is modelled as a tuple: location and orientation of the people. The \textit{discriminative detectors} used to estimate these state variables have associated noise due to various state factors such as occlusion, lighting conditions, different posture of people, motion of the robot and the people. Coupled with this, there is also stochasitcity in the state transitions, which makes it hard to compute an exact estimate of the location of the people. A principled approach to solve this problem, is to compute a \textit{belief} (posterior distribution) over the states using recursive bayesian estimation. More formally,

LeT $\textbf{X}_{t}$ be the state of the environment at time \textit{t}.
\begin{align}
P(\textbf{X}_{t} | \textbf{O}_{t}) = \dfrac{P(\textbf{O}_{t} | \textbf{X}_{t}) P(\textbf{X}_{t}|\textbf{X}_{t-1})} {P(\textbf{X}_{t-1})}
\end{align} 

Where, $P(\textbf{O}_{t} | \textbf{X}_{t})$ is the likelihood of the state given all our observations (detector outputs). Computation of this likelihood is best performed by using a learnt model of how the detectors perform for different states.

Once we compute this joint probability distribution, then the uncertainity in the location of each person can be computed by fitting gaussians around the modes of the posterior.


\subsection{Less principled approach}

Instead of computing a posterior over the states, we use a greedy algorithm for aggregating multiple detector outputs. Then we use the final output of the aggregator and learn how much uncertainity we have in the aggregator output for different state features. The state features we are interested in for this model are :
velocity of the robot, distance of the robot to each person in front of the robot. We want to learn both the probability of false detections and also probability of false negatives. 

We consider a discrete grid representation of the ground plane (grid size of 25cm*25cm). There are $\textbf{G}$ grid cells which are visible to robot sensors at any time instance. we learn the observation function over the different occupancy configurations of the grid (or features of occupancy of grid).  

Mathematically,

Let $\textbf{X}$ be the true occupancy of all the grids which are visible to the robot sensors and $\textbf{O}$ be the output of the aggregator for all the grids.

\begin{align}
P(\textbf{O}|\textbf{X}) = P(\textbf{O}|X_{i},X_{n(i)}) 
\end{align}

\begin{align}
P(\textbf{O}|X_{i},X_{n(i)}) = \prod_{i} P(O_{i}|X_{i},X_{n(i)})
\end{align}

\section{Cost map model}

\textbf{Note to self:} Cost map means final cost map unless explicitly mentioned as uncertainity cost map and social cost map.

\subsection{Ad-hoc method}

We have two un-normalized gaussians, one for social costs and one for the uncertainity. The final cost map we need is a weighted combination of both. Emperically tune the weighting of both the cost maps which gives the best trajectories.

\subsection{Learning based methods}

We are interested in the co
\subsubsection{Discriminative model}

Learn the cost map parameters using a discriminative learning algorithm. 
\begin{align}
C = \textbf{w} \Phi
\end{align}
where C is the final cost map parameters. \textbf{w} is the set of weights learnt from data. $\Phi$ is the set of features (uncertainity, social cost).


\subsubsection{Bayesian parameter estimation}

\end{document}