%In this work, we present a novel approach to human-aware navigation based on the fast marching method, by probabilistically modeling the uncertainty in on-board perception components of a social robotic system and investigating its effect on the overall social navigation performance. We have extended the model of the social costmap around a person to consider this new uncertainty factor which plays an important role in situations with noisy perception. Real robot experiments have been carried out to show the effectiveness of our approach in realistic scenarios with noisy perception involving a single dynamic human. Results show how our approach has been able to achieve trajectories which are more socially-aware compared to the basic navigation approach, and the human-aware navigation approach which relies solely on perfect perception.
In this work, we present a novel approach to human-aware navigation by probabilistically modelling the uncertainty of perception for a social robotic system and investigating its effect on the overall social navigation performance. The %common 
model of the social costmap around a person has been extended to consider this new uncertainty factor, which has been widely neglected despite playing an important role in situations with noisy perception. A social path planner based on the fast marching method has been augmented to account for the uncertainty in the positions of people.
%location used to demonstrate this idea and 
The effectiveness of the proposed approach has been tested in extensive experiments carried out with real robots and in simulation. Real experiments have been conducted, given noisy perception, in the presence of single/multiple, static/dynamic humans. Results show how this approach has been able to achieve trajectories that are able to keep a more appropriate social distance to the people, compared to those of the basic navigation approach, and the human-aware navigation approach which relies solely on perfect perception, when the complexity of the environment increases. 
%Additionally,
Accounting for uncertainty of perception is shown to result in smoother trajectories with lower jerk that are more natural from the point of view of humans. 


