\section{Conclusion and future work}
\label{sec:conclusion}

In this work we have proposed a novel approach for extending the model of social costmaps to include uncertainty of perception. Experiments show how this extended model can lead to more natural robot trajectories that preserve a social distance from people. By combining the output of a probabilistic MCMC-based tracker with an expectation costmap computation method based on convolution, we introduce a principled approach to solve the social path planning problem in real environments with multiple people while explicitly dealing with perception uncertainty. 



The idea presented in this paper can be extended to other types of perception sources. By providing probabilistic perception outputs, the proposed model can fulfill this task without loss of performance, since the probabilistic language provides a common ground to quantitatively express uncertainty regardless of the type of the sensor. 


Further improvements can be made to the accuracy of robot self-localization and the ground truth of people positions, as they have direct influence on performance evaluations. Moreover, quantifying the uncertainty of perception (investigating the impact of different levels of perception uncertainty on the behavior and performance of each part of our system) can be useful in analyzing the behavior of expectation-based social costmap computation methods for further in-depth studies.


As a future step for this research, we plan to test our approach with multiple perception sources with different uncertainties, particularly on-board perception which allows for much more flexibility in terms of applications and fewer physical limitations.% Another interesting area of exploration is replanning, \textit{i.e.}, what replanning strategy works best for a real dynamic social environment.

%Currently, replanning is done upon a change in the costmap but this is not necessarily the best strategy. It would be interesting to investigate the effect of local and global path planners along with different replanning strategies on human-aware navigation.  





%\section*{Acknowledgements}

%This work has been supported by the European MOnarCH project FP7-ICT-9-2011-601033. 


