\section{Model}
\label{sec:model}
The uncertainty based human-aware navigation system, is comprised of several key components, i.e., probabilistic perception, social path planning and navigation. In this section we will briefly explain each of the parts of our model.

\subsection{Perception}

Presence of humans in an environment should be properly perceived by a robot as a requirement for a socially-aware path planner that takes into consideration individual humans and possible social interactions taking place in the environment. This information can be obtained by an external source such as an overhead camera or can be attained using on-board sensors of the robot. Different perception sources for person detection and tracking, have different levels of uncertainty and accuracy in their detections, and are affected by various elements such as the movement of the robot, movement of the person, complexity of the environment in terms of occlusions, etc. while, there exist trackers able to perform this task with cm-level accuracy, others have a much larger uncertainty associated to them. 

Talk about deterministic v.s. probabilistic trackers. 
Say we chose omni-directional overhead camera, with which detection and tracking for both cases.Explain briefly the MCMC sampling.

We are interested in the underlying \textit{state} of the environment which is modelled as a tuple: location and orientation of the people. The \textit{discriminative detectors} used to estimate these state variables have associated noise due to various state factors such as occlusion, lighting conditions, different posture of people, motion of the robot and the people. Coupled with this, there is also stochasticity in the state transitions, which makes it hard to compute an exact estimate of the location of the people. A principled approach to solve this problem, is to compute a \textbf{belief} (posterior distribution) over the states using recursive Bayesian estimation. More formally,

Let $\textbf{X}_{t}$ be the state of the environment at time \textit{t}.
\begin{align}
P(\textbf{X}_{t} | \textbf{O}_{t}) = \dfrac{P(\textbf{O}_{t} | \textbf{X}_{t}) P(\textbf{X}_{t}|\textbf{X}_{t-1})} {P(\textbf{X}_{t-1})}
\end{align} 

Where, $P(\textbf{O}_{t} | \textbf{X}_{t})$ is the likelihood of the state given all our observations (detector outputs). Computation of this likelihood is best performed by using a learnt model of how the detectors perform for different states. Details of how these \textit{observation models} are created is explained in the next section.

$P(\textbf{X}_{t}|\textbf{X}_{t-1})$ is the \textit{transition model} which models the evolution of the state variables. For a multi person natural environment, an exact analytical model is intractable. We model the motion of the people independently. we assume a first order model for motion incorporating velocity of the person. Interactions between people are environment specific and is outside the scope of this work. So we choose to ignore this aspect of the environment and attribute it to the uncertainty we have in the state estimation. The exact model for a multi person environment is given in section ......

Another important aspect is the computation of this belief itself. the state space is extremely complex so as to compute exactly this probability 
distribution over the states. We use an MCMC based sampling algorithm to approximately compute the belief. The details of our sampling algorithm is given in section .......

%For the purpose of the human aware navigation planner, we transform these beliefs to a lower dimensional representation which is a 2-dimensional gaussian over the location of the person.

\subsubsection{Detector}
We use a probabilistic background based detector
\subsubsection{Observation models}
We learn from data, the distribution $P(\textbf{O}|\textbf{X})$ for each detector.








%measures:paper Including Human Factors for Planning Comfortable Paths: Bio-sensing devices , modelling passenger comfort in a real environment to build a path planner. 
%Visibility index: Y. Morales, J. Even, N. Kallakuri, T. Ikeda, K. Shinozawa, T. Kondo, and N. Hagita, “Visibility analysis for autonomous vehicle comfortable navigation,” in Robotics and Automation (ICRA), 2014 IEEE Interna- tional Conference on, May 2014, pp. 2197–2202