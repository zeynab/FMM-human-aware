\section{perception model}
\label{perception_model}
%The uncertainity based human-aware navigation system, is comprised of several key components, i.e., probabilistic perception, social path planning and navigation. In this section we will briefly explain each of the parts of our model.


Presence of humans in an environment should be properly perceived by a robot. A socially-aware path planner needs to take into consideration individual humans and possible social interactions taking place.% in the environment. 
~This information can be obtained by an external source such as an overhead camera or can be attained using on-board sensors of the robot. Different perception sources for person detection and tracking, have different levels of uncertainty and accuracy in their detections, and are affected by components such as the movement of the robot, movement of the person, complexity of the environment in terms of occlusions, etc. By taking the uncertainty of perception into account in a human-aware path planner, the same planning method could easily be reused even when the source of perception changes. 

In the proposed framework, a probabilistic approach has been chosen to account for uncertainty in the tracking rather than deterministically reporting positions and dealing with the perception uncertainty inside the tracker module only. We are interested in the underlying \textit{state} of the environment which is the position of the people. The \textit{detectors} used to estimate these state variables have associated noise due to various factors such as occlusion, lighting conditions, different postures of people, motion of the robot and the people, etc. Coupled with this, there is also stochasticity in the state transitions, which makes it hard to compute an exact estimate of the location of the people. A principled approach to solve this problem, is to compute a \textit{belief} (posterior distribution) over the states using recursive Bayesian estimation. We first describe the state representation of the system and then explain the tracking model formally.

\subsection{Detector}
The background-based detector proposed in~\cite{englebienne-bnaic} is a very effective probabilistic method, which allows the automatic evaluation of the number of people in the scene and detection of their locations. This method has the following advantages: (1) It can incorporate prior knowledge, including which areas in the scene can contain people and how probable it is for people to be in those locations; a probability distribution over the number of people in the scene; a probabilistic model of how close
together people tend to walk; etc. (2) The complexity of the algorithm depends linearly on the number of people in the scene.% When many people are present in the frame, detecting all of them requires
%more than 40 \textit{ms} with our current implementation of the algorithm. Although it still requires far less than a second, further optimizations could easily improve this performance. 
~(3) The method is robust to changes in illumination, shadows and occlusions. It is adapted it to adjust to a non-static background automatically.

\subsection{State Representation}

An occupancy grid-based approach is used for tracking the people in the environment.% in this work. 
~The environment is discretized into $G$ cells. Each cell is of size 25 \textit{cm} by 25 \textit{cm}. The size of cells have been chosen in such a way that each cell can be occupied at most with one person at any time. The occupancy of each cell is denoted by $X_{i}$ where $i \in G$. The occupancy of all the cells at time $t$ is the state of the world $\textbf{X}_{t}$. At every time instance $t$, the observations from the detector for each cell $i$ is given by $O_{i}$. The set of observations for the whole state is denoted by $\textbf{O}_{t}$. 

\subsection{Tracking Model}

Let $\textbf{X}_{t}$ be the state of the environment at time \textit{t}. We are interested in computing the current belief over the states $\textbf{X}_{t}$ given the observations $\textbf{O}_{1:t}$. Formally this can be represented by a Bayes filter~\cite{khan2004mcmc} as,

%\begin{align}
%P(\textbf{X}_{t} | \textbf{O}_{t},\textbf{X}_{t-1}) = \dfrac{P(\textbf{O}_{t} | \textbf{X}_{t},\textbf{X}_{t-1}) P(\textbf{X}_{t}|\textbf{X}_{t-1})} {P(\textbf{O}_{t}|\textbf{X}_{t-1})}
%\end{align} 

\begin{align}
P(\textbf{X}_{t} | \textbf{O}_{1:t}) \propto   P(\textbf{O}_{t} | \textbf{X}_{t}) \sum\limits_{\textbf{X}_{t-1}} P(\textbf{X}_{t}|\textbf{X}_{t-1}) P(\textbf{X}_{t-1}|\textbf{O}_{1:t-1}) \end{align} 

%Observations at time \textit{t} only depends on the state at time \textit{t}.
%Now, $\textbf{O}_{t}$ only depends on $\textbf{X}_{t}$ and not on $\textbf{X}_{t-1}$. Applying this in equation above,

%\begin{align}
%P(\textbf{X}_{t} | \textbf{O}_{t},\textbf{X}_{t-1}) = \dfrac{P(\textbf{O}_{t} | \textbf{X}_{t}) P(\textbf{X}_{t}|\textbf{X}_{t-1})} {P(\textbf{O}_{t})}
%\end{align} 


Where, $P(\textbf{O}_{t} | \textbf{X}_{t})$ is the likelihood of the state given all our observations (detector outputs). Computation of this likelihood is best performed by using a learned model of how the detectors perform in different states.

$P(\textbf{X}_{t}|\textbf{X}_{t-1})$ is the \textit{transition model} which models the evolution of the state variables. For a real multi person environment, an exact analytical model is intractable. In our system, we use a simple uniform distribution for the transition model. We assume that people move randomly and that there is an equal probability of motion in any direction. 

$P(\textbf{X}_{t-1}|\textbf{O}_{1:t-1})$ is the posterior probability computed from the previous time step.

In a multi person environment, the state space is extremely complex for computing the exact probability 
distribution over the states. We use an MCMC-based sampling algorithm to approximately compute the belief. In the next section the implementation details of our probabilistic model for person detection and tracking are explained.


Although the detector is modelled probabilistically, it is still needed to learn the distribution $P(\textbf{O}|\textbf{X})$ for the detector from data. Given the labelled location of the people in a data set, the uncertainty in the observations is learned for all the locations and configurations of the state space. Since the state space has high dimensionality, we learn the likelihood model over a parameterised state space \cite{berclaz-fleuret-fua-2008}.



\subsection{MCMC Sampling}
\label{MCMC}
Markov Chain Monte Carlo is a widely used sampling algorithm for estimation of complex posterior distribution. It has been gaining popularity in multi-target tracking applications \cite{khan2004mcmc}. Compared to traditional particle filters, MCMC-based sampling leads to far less sample impoverishment%[ref]
~and thus a much better estimate of the state over time. The core idea of MCMC is to generate samples from a Markov chain. The samples are then evaluated using a \textit{proposal distribution} and accepted or rejected based on an \textit{acceptance ratio}. The proposal distribution should be proportional to the posterior distribution that we are trying to approximate. The MCMC sampler creates hypothesis of the location of the people in the grids. Each sample is an estimate of the occupancy of all the cells taken collectively. 

In this work, the occupancy of the grids are used as hypothesis. Each cell can either be occupied or not, with initially starting from a random distribution of occupancy and then generating samples using the following moves:

1. \textit{Birth-Death proposals:}
A cell is randomly selected, and the sample state of the cell is flipped. If the cell was occupied, a proposal which makes the cell unoccupied is generated and vice-versa.

2. \textit{Move proposals:}
In this case, an occupied cell is selected and the occupancy is randomly moved to one of the 8 connecting neighbors.

Once the proposal sample is generated, we evaluate the original sample and the proposed sample with reference to a proposal distribution. In our case, we use a learned observation model of the detector output as the proposal distribution. We fold in the detector output $\textbf{O}_{t}$ while evaluating the proposals using the proposal distribution. Every proposal is a hypothesis of the state $\textbf{X}_{t}$.
Evaluating the proposals will give us an acceptance ratio. If the acceptance ratio is greater than 1 the sample is accepted unconditionally. 
%If the acceptance probability is less than 1
Otherwise, we randomly sample from a uniform distribution and then accept or reject the sample if the acceptance ratio is greater than the sampled value of uniform distribution. Formally,
The acceptance ratio is computed as:
\begin{align}
Acc(x|x^{'}) = \min\Big\lbrace\Big(\dfrac{P(\textbf{O}_{t}|x)\sum\limits_{x_{t-1}} P(x|x_{t-1})}{P(\textbf{O}_{t}|x^{'})\sum\limits_{x_{t-1}} P(x^{'}|x_{t-1})}\Big),1\Big\rbrace
\end{align}
Where $x$ is the proposed sample of state $\textbf{X}_{t}$, and $x^{'}$ is the initial sample. $x_{t-1}$ is a sample of the state $\textbf{X}_{t-1}$.

If the sample is accepted, the currently proposed sample will be used as the initial sample for the next step of the MCMC sampling. If rejected, the sample is still added to the set of hypothesis, but we start sampling again from the old sample.


Sampling is successively repeated until $N_{s}$ samples are accepted. The threshold for the number of accepted samples are set to be 100 in our experiments. Once $N_{s}$ samples are accepted, they are collectively represented as the approximate representation of the multi modal posterior distribution. 

For the purpose of social navigation, these samples are converted to a set of particles. Even though the set of samples are fixed, the set of particles can vary, since each sample is a joint sample which represents the whole occupancy grid. When there are multiple people in the environment, each one of the samples can be decomposed into a set of particles, proportional to the number of people in the environment. %We represent the set of particles by $N_{p}$. 