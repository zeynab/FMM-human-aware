\section{Introduction}


Robots will progressively become part of the work spaces and habitats of humans.
Despite where they are located, in home environments or hospitals as assistants, in factories as co-workers or as guides in supermarkets, a key behavior which they all share is navigation.
Robots have to navigate in environments shared with humans and the quality of their movement strongly influences  how their intelligence is perceived~\cite{Althaus2004}. Consequently, new methods for robot navigation in the presence of humans are being studied extensively. Kruse et al.~\cite{Kruse2013} present a thorough survey of such methods where they identify comfort, naturalness and scalability as the three key issues addressed by the existing human-aware robot navigation methods so far. %(last 2 sentences taken from  RV's paper)
%Hence socially-aware navigation which pays special consideration to people 

Presence of humans in an environment should be properly perceived by a robot as a requirement for a socially-aware path planner that takes into consideration individual humans and possible social interactions taking place in the environment. This information can be obtained by an external source such as an overhead camera or can be attained using on-board sensors of the robot. while there are advantages and disadvantage to both cases, on-board perception allows for a fully autonomous and independent robot that relies only on itself for taking decisions about the world. Moreover, there are environments which do not permit the use of external infrastructures required for providing the requisite perception, due to reasons such as privacy or security issues or etc. Therefore, providing the required perception for human-aware navigation with on-board sensors is a desirable characteristic for a robot which helps integration of such systems into real-world environments.  


However, perception will never be perfect and is affected by various elements such as the movement of the robot, movement of the person, complexity of the environment in terms of occlusions, etc. Unlike other works which choose to ignore this very important aspect of the uncertainty in perception, we propose a model which computes the uncertainty of the location and orientation of the people in an environment. We aim to study, how this factor should influences the social costs used by the path planner and how taking this into account the resulting trajectories will be improved in terms of social acceptability.

One important concept which is used in numerous studies~\cite{Mumm2011,Takayama2009,Walters2011,ferrer2013robot} in this area is virtual space around a person that is mutually respected by other humans, called \textit{proxemics}~\cite{Hall1969}.
Based on this concept, depending on the relationship and the interaction that exists between humans, people choose different social distances relating to intimate, personal, social or public contexts.
Changes in the expected distance may indicate dislike if it is too large or cause discomfort if it is too small.



As stated in~\cite{gomez2013social}, recent work in the human-robot interaction domain can be divided in three categories of 1) human-robot proxemics, 2) human-aware planning and navigation, and 3) robot-to-human approaching and behaving. In this work we focus mostly on the second category using the concepts from human-robot proxemics. %Human-aware path planning has been an area of interest particularly in the recent years. 

In this research we are introduce a social path planning approach based on the fast marching method (FMM)~\cite{sethian1999fast} which has been proven to be successful in real domestic spaces \cite{ventura2015}. However, we add a social component to the aforementioned method by augmenting it with social costmaps \textemdash based on proxemics principles~\cite{kirby2009companion}\textemdash~around individual people which correspond to speedmaps for the FMM method. There have previously been a number of research papers which have address social path planning~\cite{gomez2014fast,gomez2013social} using FMM. 

The work of ~\cite{gomez2014fast} a theoretical framework for introduced sub-problems of social path planning is presented and an extended mode for engaging groups of people is proposed by using a special version of fast marching square planning method~\cite{valero2013fast}. Nonetheless, the information about humans are considered to be given and noiseless, while only simulations have been used to show the effectiveness of the proposed method for static people. We are interested in investigating the same problem in real-world scenarios with the challenges that exist therein. Particularly, in the case of moving people, while the perception is obtained only by the use of on-board resources of the robot and is consequently subject to different levels of uncertainty. 
%We believe that uncertainty of perception in social robotics is an important topic which to our knowledge is very little explored.
We believe that uncertainty of perception in social robotics, is an important topic which to our knowledge, has not been the subject of notable explorations. There is a dedicated chapter in \cite{correa2014uncertainty} on local planning with uncertainty, however this is not considered in a social context. The sources of uncertainty in~\cite{correa2014uncertainty} are the position of the robot and the obstacles, and the partially known motion of moving obstacles.


The remainder of this article is organized as follows. In section~\ref{sec:background} we introduce the background for this work. In section~\ref{sec:results}, we show the results of real robot experiments and finally, section~\ref{sec:conclusion} concludes this work.

%\paragraph{Story}  - Quantify uncertainity in perception. Use this for generating cost maps adaptively. FMM creates plans using this uncertainity model.





% \paragraph{Motivation and approach} - Perception will never be perfect. Unlike other works which choose to ignore this very important aspect of the uncertainty in perception, we propose a model which computes the uncertainty of the location and orientation of multiple people in an environment. We use this model to make informed choices about the cost function for navigation which is then used by a state of the art navigation model using FMM. We dynamically replan based on the perception information available. The key idea is that in cases where the perception is unreliable, we need to make sure that the robot navigates prioritizing safety over social norms.

% paragraph{Note} Active perception is also important. The robot movement causes uncertainty which can be actively reduced by taking paths which improve the perception information.
