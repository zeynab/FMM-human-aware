\section{Background}
\label{sec:background}
Human-aware navigation focuses on the interaction dynamics between humans and robots that occur as a result of navigation~\cite{Kruse2013}. In the literature, we can find several strategies for comfort ranging from appropriate approaching strategy~\cite{Dautenhahn2006}, maintaining appropriate distance~\cite{Takayama2009}, control strategies to avoid being noisy~\cite{Martinson2007} and use of planning for avoiding interference~\cite{Vasquez2012}. Additionally, several approaches to generate legibility for robot navigation, have been reported in \cite{lichtenthaler2013towards}.

Our goal is to show that the assumption of having perfect information about the state of the human is unrealistic and in real situations when the robot has to rely on its on-board perception to get the information required for human-aware path planning, the uncertainty in this information can not be ignored. We can think of false negative detections where the robot misses to take on person into consideration, false positive detections where other objects are detected as humans, noisy estimations of position, orientation, velocity of the person, etc. as examples of situations with perception uncertainty that demand special attention when dealing with humans and in more complex scenarios involving group interactions, this becomes even more important. So human-aware path planning methods need to account for this factor which has not received much attention.

We base our navigation method on FMM \cite{ventura2015} for path planning and DWA~\cite{fox1997dynamic} for obstacle avoidance. On-board perception in obtained through the use of a Kinect RGB-D camera and a laser scanner mounted on the robot. We will explain our robotic platform~\cite{Messias2014robotic} and the experimental setup in detail in section \ref{sec:experimental_setup}. We adopt the social space modeling already used by \cite{kirby2009companion,gomez2014fast} and augment it with a new feature representing perception uncertainty for human detection and tracking.


measures:

paper Including Human Factors for Planning Comfortable Paths:
Bio-sensing devices 
modeling passenger comfort in a real environment to build a path planner.



Visibility index:
Y. Morales, J. Even, N. Kallakuri, T. Ikeda, K. Shinozawa, T. Kondo, and N. Hagita, “Visibility analysis for autonomous vehicle comfortable navigation,” in Robotics and Automation (ICRA), 2014 IEEE Interna- tional Conference on, May 2014, pp. 2197–2202